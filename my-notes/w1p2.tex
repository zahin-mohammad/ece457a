\documentclass{article}
\usepackage[utf8]{inputenc}
\usepackage{biblatex}
\usepackage{amssymb}
\usepackage{amsmath}
\usepackage{marginnote}
\usepackage{hyperref}


\addbibresource{bib.bib}
\setlength{\parindent}{0em}
\bibliography{bib}


\title{Week 1 Part 2}
\author{Zahin Mohammad}
\date{May 2020}
\linespread{1.5}
\begin{document}

\maketitle
% \setlength{\parskip}{6pt}
\section{Artificial Intelligence}
Goals of AI:
\begin{itemize}
    \item Create expert systems (learn, demonstrate, explain and advice users) 
    \item Implement human intelligence (understand, think, learn and behave like humans)
\end{itemize}

System Categorization:
\begin{itemize}
    \item Hypothesis / experimentation
    \item \begin{itemize}
        \item Thinks like humans
        \item Acts like humans
    \end{itemize}
    \item Mathematics / Engineering
    \item \begin{itemize}
        \item Thinks rationally
        \item Behaves rationally
    \end{itemize}
\end{itemize}

Rational Systems:
\begin{itemize}
    \item Uses logic to achieve goals
    \item Hard to represent informal knowledge
    \item Act to achieve some goal (not always correct, but will accomplish task)
\end{itemize}

Agents:
\begin{itemize}
    \item We create the agent(something that senses the environment and acts on it)
    \item Agents apply actions to the environment via actuators
    \item Agents get information from the environment via sensors
    \item A rational agent will act in a way to to maximize performance (expected performance) based on history and built-in knowledge
\end{itemize}

Adaption:
\begin{itemize}
    \item Adjust and improve behavior
    \item Learn via instructor, example, discovery
    \item Generalization
\end{itemize}

Cooperation:
\begin{itemize}
    \item Direct 
    \item Indirect
    \item Independent
\end{itemize}

\printbibliography[title={Referências}]
\end{document}
