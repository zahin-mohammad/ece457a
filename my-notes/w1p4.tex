\documentclass{article}
\usepackage[utf8]{inputenc}
\usepackage{biblatex}
\usepackage{amssymb}
\usepackage{amsmath}
\usepackage{marginnote}
\usepackage{hyperref}


\addbibresource{bib.bib}
\setlength{\parindent}{0em}
\bibliography{bib}


\title{Week 1 Part 4}
\author{Zahin Mohammad}
\date{May 2020}
\linespread{1.5}
\begin{document}

\maketitle
% \setlength{\parskip}{6pt}
\section{Solution Strategies for Ill-Structured Problems}
    Optimization Problem: Find the best solution from feasible solutions subject to constraints.

    Optimization Algorithms:
    \begin{itemize}
        \item Exact Algos:
        \item  \begin{itemize}
            \item Optimal Solution
            \item High computational cost
        \end{itemize}
        \item Approximate Algos:
        \item \begin{itemize}
            \item Near-Optimal solutions
            \item Low computational cost
        \end{itemize}
    \end{itemize}

\section{Optimization Methods}
\subsection{Approximate Algorithms}
    Heuristics:
    \begin{itemize}
        \item Produces acceptable solutions to complex problems in a practical time
        \item Aim is to efficiently generate good solutions
        \item Does not guarantee optimality
        \item Characteristics:
        \item \begin{itemize}
            \item Short run times
            \item Easy to implement
            \item Flexible
            \item Simple
        \end{itemize}
    \end{itemize}
    
    Constructive Methods:
    \begin{itemize}
        \item Start from scratch
        \item Build solution one component at a time
    \end{itemize}
    Local Search Methods:
    \begin{itemize}
        \item Start from initial solution
        \item Iteratively try and replace current solution with a better one
    \end{itemize}
\subsection{Local Search Methods: In depth}
    Search: move from state-to-state in the problem space to find a goal or terminate if not found.

    Goal Based Search:
    \begin{itemize}
        \item Use information about environment and current state to reach an end goal
        \item Goal can be sub-optimal
        \item Find a path from A to B
    \end{itemize}
    Utility Based Search:
    \begin{itemize}
        \item Same as goal based search but considers cost
        \item Shortest path from A to B
    \end{itemize}

    Requirements for search:
    \begin{itemize}
        \item Goal Formation
        \item Problem Formation
    \end{itemize}

    Goal Formation:
    \begin{itemize}
        \item What aspects are we interested in?
        \item What aspects can be ignored?
    \end{itemize}

    Properties of Search Algorithms:
    \begin{itemize}
        \item Completeness: Are we guaranteed to find a goal node if it exists
        \item Optimality Are we guaranteed to find the best goal node
        \item Time Complexity: How many nodes are generated
        \item Space Complexity: Max number of nodes stored in memory
    \end{itemize}

    Problem Formulation:
    \begin{itemize}
        \item Decide how to manipulate important aspects
        \item Done after goal formulation
        \item Compact representation of problem space
        \item Represent the search space by states 
        \item Need valid actions for a given state:
        \begin{itemize}
            \item Define action the agent can perform and their cost
            \item Transition model (what decides the state-state transitions; why would the agent want to go to the new state)
            \item Neighbor of a state are states the agent can transition too
        \end{itemize}
        \item Well Defined state-space formulation:
        \begin{itemize}
            \item State Space: Complete / partial representation of a problem 
            \item Initial State: Starting point
            \item Goal State: Search termination state
            \item Set of Actions: Allows movement between Strategies
            \item Cost
        \end{itemize}
    \end{itemize}

    Define the Solution:
    \begin{itemize}
        \item A sequence of actions that will transform the environment from one state to another to reach the goal
    \end{itemize}

    Closed world: Each state is a complete description of the world.
    \setlength{\parskip}{6pt}
    
    Environment is:
    \begin{itemize}
        \item Full observable (can measure environment after an action)
        \item Deterministic (action will product same results)
        \item Sequential
        \item Static
        \item Discrete (move from one state to the next)
    \end{itemize}

    Template of Generic Search:
    \begin{itemize}
        \item Choose, Test, Expand and representation
        \item Search algo's differ in how next state is chosen (greedy, heuristic, etc)
        \item Generally queue used to store fringe nodes
    \end{itemize}

\printbibliography[title={Referências}]
\end{document}
